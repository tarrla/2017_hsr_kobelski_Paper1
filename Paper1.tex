\documentclass{aastex62}

%% The default is a single spaced, 10 point font, single spaced article.
%% There are 5 other style options available via an optional argument. They
%% can be envoked like this:
%%
%% \documentclass[argument]{aastex62}
%% 
%% where the layout options are:
%%
%%  twocolumn   : two text columns, 10 point font, single spaced article.
%%                This is the most compact and represent the final published
%%                derived PDF copy of the accepted manuscript from the publisher
%%  manuscript  : one text column, 12 point font, double spaced article.
%%  preprint    : one text column, 12 point font, single spaced article.  
%%  preprint2   : two text columns, 12 point font, single spaced article.
%%  modern      : a stylish, single text column, 12 point font, article with
%% 		  wider left and right margins. This uses the Daniel
%% 		  Foreman-Mackey and David Hogg design.
%%  RNAAS       : Preferred style for Research Notes which are by design 
%%                lacking an abstract and brief. DO NOT use \begin{abstract}
%%                and \end{abstract} with this style.
%%
%% Note that you can submit to the AAS Journals in any of these 6 styles.
%%
%% There are other optional arguments one can envoke to allow other stylistic
%% actions. The available options are:
%%
%%  astrosymb    : Loads Astrosymb font and define \astrocommands. 
%%  tighten      : Makes baselineskip slightly smaller, only works with 
%%                 the twocolumn substyle.
%%  times        : uses times font instead of the default
%%  linenumbers  : turn on lineno package.
%%  trackchanges : required to see the revision mark up and print its output
%%  longauthor   : Do not use the more compressed footnote style (default) for 
%%                 the author/collaboration/affiliations. Instead print all
%%                 affiliation information after each name. Creates a much
%%                 long author list but may be desirable for short author papers
%%
%% these can be used in any combination, e.g.
%%
%% \documentclass[twocolumn,linenumbers,trackchanges]{aastex62}
%%
%% AASTeX v6.* now includes \hyperref support. While we have built in specific
%% defaults into the classfile you can manually override them with the
%% \hypersetup command. For example,
%%
%%\hypersetup{linkcolor=red,citecolor=green,filecolor=cyan,urlcolor=magenta}
%%
%% will change the color of the internal links to red, the links to the
%% bibliography to green, the file links to cyan, and the external links to
%% magenta. Additional information on \hyperref options can be found here:
%% https://www.tug.org/applications/hyperref/manual.html#x1-40003
%%
%% If you want to create your own macros, you can do so
%% using \newcommand. Your macros should appear before
%% the \begin{document} command.
%%
\newcommand{\vdag}{(v)^\dagger}
\newcommand\aastex{AAS\TeX}
\newcommand\latex{La\TeX}

%% Tells LaTeX to search for image files in the 
%% current directory as well as in the figures/ folder.
\graphicspath{{./}{figures/}}

%% Reintroduced the \received and \accepted commands from AASTeX v5.2
\received{January 1, 2018}
\revised{January 7, 2018}
\accepted{\today}
%% Command to document which AAS Journal the manuscript was submitted to.
%% Adds "Submitted to " the arguement.
\submitjournal{ApJ}

%% Mark up commands to limit the number of authors on the front page.
%% Note that in AASTeX v6.2 a \collaboration call (see below) counts as
%% an author in this case.
%
%\AuthorCollaborationLimit=3
%
%% Will only show Schwarz, Muench and "the AAS Journals Data Scientist 
%% collaboration" on the front page of this example manuscript.
%%
%% Note that all of the author will be shown in the published article.
%% This feature is meant to be used prior to acceptance to make the
%% front end of a long author article more manageable. Please do not use
%% this functionality for manuscripts with less than 20 authors. Conversely,
%% please do use this when the number of authors exceeds 40.
%%
%% Use \allauthors at the manuscript end to show the full author list.
%% This command should only be used with \AuthorCollaborationLimit is used.

%% The following command can be used to set the latex table counters.  It
%% is needed in this document because it uses a mix of latex tabular and
%% AASTeX deluxetables.  In general it should not be needed.
%\setcounter{table}{1}

%%%%%%%%%%%%%%%%%%%%%%%%%%%%%%%%%%%%%%%%%%%%%%%%%%%%%%%%%%%%%%%%%%%%%%%%%%%%%%%%
%%
%% The following section outlines numerous optional output that
%% can be displayed in the front matter or as running meta-data.
%%
%% If you wish, you may supply running head information, although
%% this information may be modified by the editorial offices.
\shorttitle{The Kolsch of Data Sets}
\shortauthors{Kobelski et al.}
%%
%% You can add a light gray and diagonal water-mark to the first page 
%% with this command:
% \watermark{text}
%% where "text", e.g. DRAFT, is the text to appear.  If the text is 
%% long you can control the water-mark size with:
%  \setwatermarkfontsize{dimension}
%% where dimension is any recognized LaTeX dimension, e.g. pt, in, etc.
%%
%%%%%%%%%%%%%%%%%%%%%%%%%%%%%%%%%%%%%%%%%%%%%%%%%%%%%%%%%%%%%%%%%%%%%%%%%%%%%%%%

%% This is the end of the preamble.  Indicate the beginning of the
%% manuscript itself with \begin{document}.

\begin{document}

\title{Transient brightenings observed with multiple instruments}
\correspondingauthor{Adam R. Kobelski}
\email{adam.kobelski@mail.wvu.edu, ltarr2@gmu.edu, sjaeggli@nso.edu, sabrina.savage@nasa.gov}

\author{Adam R. Kobelski}
\affiliation{West Virginia University}

\author[0000-0002-8259-8303]{Lucas A. Tarr}
\affiliation{George Mason University}

\author{Sarah Jaeggli}
\affiliation{National Solar Observatory}

%% The \author command can take an optional ORCID.
\author{Sabrina Savage}
\affiliation{NASA Marshall Space Flight Center}
\begin{abstract}
Here is our abstract.  We did work

\end{abstract}
\keywords{The Sun, Radio, XRay, EUV}

\section{Introduction} \label{sec:intro}
Our work is awesome, here's why
\section{Observations} \label{sec:Obs}
\subsection{IBIS}
The IBIS data was reduced primarily using the pipeline code provided by NSO\footnote{Version 1.4, available here: https://www.nso.edu/telescopes/dunn-solar-telescope/dst-pipelines/}

\subsection{ALMA}

\subsection{Hinode}

\section{Coalignment} \label{sec:coalign}
Coaligning the various instruments was a challenge.  We did it like this:

\startlongtable
\begin{deluxetable}{c|cc}
\tablecaption{ApJ costs from 1991 to 2013\tablenotemark{a} \label{tab:table}}
\tablehead{
\colhead{Year} & \colhead{Subscription} & \colhead{Publication} \\
\colhead{} & \colhead{cost} & \colhead{charges\tablenotemark{b}}\\
\colhead{} & \colhead{(\$)} & \colhead{(\$/page)}
}
\colnumbers
\startdata
1991 & 600 & 100 \\
1992 & 650 & 105 \\
1993 & 550 & 103 \\
1994 & 450 & 110 \\
1995 & 410 & 112 \\
1996 & 400 & 114 \\
1997 & 525 & 115 \\
1998 & 590 & 116 \\
1999 & 575 & 115 \\
2000 & 450 & 103 \\
2001 & 490 &  90 \\
2002 & 500 &  88 \\
2003 & 450 &  90 \\
2004 & 460 &  88 \\
2005 & 440 &  79 \\
2006 & 350 &  77 \\
2007 & 325 &  70 \\
2008 & 320 &  65 \\
2009 & 190 &  68 \\
2010 & 280 &  70 \\
2011 & 275 &  68 \\
2012 & 150 &  56 \\
2013 & 140 &  55 \\
\enddata
\tablenotetext{a}{Adjusted for inflation}
\tablenotetext{b}{Accounts for the change from page charges to digital quanta in April, 2011}
\tablecomments{Note that {\tt \string \colnumbers} does not work with the 
vertical line alignment token. If you want vertical lines in the headers you
can not use this command at this time.}
\end{deluxetable}


\section{Displaying mathematics} \label{sec:displaymath}



\section{Revision tracking and color highlighting} \label{sec:highlight}

Authors sometimes use color to highlight changes to their manuscript in
response to editor and referee comments.  In \aastex\ new commands
have been introduced to make this easier and formalize the process. 

The first method is through a new set of editing mark up commands that
specifically identify what has been changed.  These commands are
{\tt\string\added\{<text>\}}, {\tt\string\deleted\{<text>\}}, and
{\tt\string\replaced\{<old text>\}\{<replaced text>\}}. To activate these
commands the {\tt\string trackchanges} option must be used in the
{\tt\string\documentclass} call.  When compiled this will produce the
marked text in red.  The {\tt\string\explain\{<text>\}} can be used to add
text to provide information to the reader describing the change.  Its
output is purple italic font. To see how {\tt\string\added\{<important
added info>\}}, {\tt\string\deleted\{<this can be deleted text>\}},
{\tt\string\replaced\{<old data>\}\{<replaced data>\}}, and \break
{\tt\string\explain\{<text explaining the change>\}} commands will produce
\added{important added information}\deleted{, deleted text, and }
\replaced{old data}{and replaced data,} toggle between versions compiled with
and without the {\tt\string trackchanges} option.\explain{text explaining
the change}

A summary list of all these tracking commands can be produced at the end of
the article by adding the {\tt\string\listofchanges} just before the
{\tt\string\end\{document\}} call.  The page number for each change will be
provided. If the {\tt\string linenumbers} option is also included in the
documentcall call then not only will all the lines in the article be
numbered for handy reference but the summary list will also include the
line number for each change.

The second method does not have the ability to highlight the specific
nature of the changes but does allow the author to document changes over
multiple revisions.  The commands are {\tt\string\edit1\{<text>\}},
{\tt\string\edit2\{<text>\}} and {\tt\string\edit3\{<text>\}} and they
produce {\tt\string<text>} that is highlighted in bold red, italic blue and
underlined purple, respectively.  Authors should use the first command to
\edit1{indicated which text has been changed from the first revision.}  The
second command is to highlight \edit2{new or modified text from a second
revision}.  If a third revision is needed then the last command should be used 
\edit3{to show this changed text}.  Since over 90\% of all manuscripts are
accepted after the 3rd revision these commands make it easy to identify
what text has been added and when.  Once the article is accepted all the
highlight color can be turned off simply by adding the
{\tt\string\turnoffediting} command in the preamble. Likewise, the new commands
{\tt\string\turnoffeditone}, {\tt\string\turnoffedittwo}, and
{\tt\string\turnoffeditthree} can be used to only turn off the 
{\tt\string\edit1\{<text>\}}, {\tt\string\edit2\{<text>\}} and 
{\tt\string\edit3\{<text>\}}, respectively.

Similar to marking editing changes with the {\tt\string\edit} options there
are also the {\tt\string\authorcomments1\{<text>\}}, 
{\tt\string\authorcomments2\{<text>\}} and
{\tt\string\authorcomments3\{<text>\}} commands.  These produce the same
bold red, italic blue and underlined purple text but when the
{\tt\string\turnoffediting} command is present the {\tt\string<text>}
material does not appear in the manuscript.  Authors can use these commands
to mark up text that they are not sure should appear in the final
manuscript or as a way to communicate comments between co-authors when
writing the article.

\section{Software and third party data repository citations} \label{sec:cite}

The AAS Journals would like to encourage authors to change software and
third party data repository references from the current standard of a
footnote to a first class citation in the bibliography.  As a bibliographic
citation these important references will be more easily captured and credit
will be given to the appropriate people.

The first step to making this happen is to have the data or software in
a long term repository that has made these items available via a persistent
identifier like a Digital Object Identifier (DOI).  A list of repositories
that satisfy this criteria plus each one's pros and cons are given at \break
\url{https://github.com/AASJournals/Tutorials/tree/master/Repositories}.

In the bibliography the format for data or code follows this format: \\

\noindent author year, title, version, publisher, prefix:identifier\\



%% If you wish to include an acknowledgments section in your paper,
%% separate it off from the body of the text using the \acknowledgments
%% command.
\acknowledgments

We thank all the people that have made this AASTeX what it is today.  This
includes but not limited to Bob Hanisch, Chris Biemesderfer, Lee Brotzman,
Pierre Landau, Arthur Ogawa, Maxim Markevitch, Alexey Vikhlinin and Amy
Hendrickson. Also special thanks to David Hogg and Daniel Foreman-Mackey
for the new "modern" style design. Considerable help was provided via bug
reports and hacks from numerous people including Patricio Cubillos, Alex
Drlica-Wagner, Sean Lake, Michele Bannister, Peter Williams, and Jonathan
Gagne.

%% To help institutions obtain information on the effectiveness of their 
%% telescopes the AAS Journals has created a group of keywords for telescope 
%% facilities.
%
%% Following the acknowledgments section, use the following syntax and the
%% \facility{} or \facilities{} macros to list the keywords of facilities used 
%% in the research for the paper.  Each keyword is check against the master 
%% list during copy editing.  Individual instruments can be provided in 
%% parentheses, after the keyword, but they are not verified.

\vspace{5mm}
\facilities{HST(STIS), Swift(XRT and UVOT), AAVSO, CTIO:1.3m,
CTIO:1.5m,CXO}

%% Similar to \facility{}, there is the optional \software command to allow 
%% authors a place to specify which programs were used during the creation of 
%% the manusscript. Authors should list each code and include either a
%% citation or url to the code inside ()s when available.

\software{astropy \citep{2013A&A...558A..33A}
          }

%% Appendix material should be preceded with a single \appendix command.
%% There should be a \section command for each appendix. Mark appendix
%% subsections with the same markup you use in the main body of the paper.

%% Each Appendix (indicated with \section) will be lettered A, B, C, etc.
%% The equation counter will reset when it encounters the \appendix
%% command and will number appendix equations (A1), (A2), etc. The
%% Figure and Table counter will not reset.

\appendix

\section{Appendix information}


\begin{thebibliography}{}

\bibitem[Astropy Collaboration et al.(2013)]{2013A&A...558A..33A} Astropy Collaboration, Robitaille, T.~P., Tollerud, E.~J., et al.\ 2013, \aap, 558, A33 

\end{thebibliography}

%% This command is needed to show the entire author+affilation list when
%% the collaboration and author truncation commands are used.  It has to
%% go at the end of the manuscript.
%\allauthors

%% Include this line if you are using the \added, \replaced, \deleted
%% commands to see a summary list of all changes at the end of the article.
%\listofchanges

\end{document}

% End of file `sample62.tex'.
